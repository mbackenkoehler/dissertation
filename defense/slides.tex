\documentclass{beamer}
\usetheme{default}


\setbeamertemplate{footline}[frame number]

\title{Analysis of Markovian Population Models}
\subtitle{Dissertation Defense}
\author{Michael Backenk\"{o}hler}
\institute{Saarland Informatics Campus}
\date{September 20, 2022}

\begin{document}

\begin{frame}
\titlepage
\end{frame}

\begin{frame}{Motivation}
    This is your first presentation!
\end{frame}

\begin{frame}{Markovian Population Models}{Semantics}
  \begin{itemize}
    \item counting agents / population size
    \item continuous time
    \item exponential jump times / CTMC dynamics
  \end{itemize}
\end{frame}

\begin{frame}{Markovian Population Models}{Stationary Distribution}
\end{frame}

\begin{frame}{Markovian Population Models}{Moment Dynamics}
  \begin{itemize}
    \item Moment formula
    \item Moment figure (distribution vs. Moments and SSA)
    \item Analytic integration and resulting martingale process
  \end{itemize}
\end{frame}

\begin{frame}{Bounding Mean First-Passage Times}{Martingale Process and Linear Moment Constraints}
  \begin{itemize}
      \item expected occupation time and exit measures (in relation to expectation of the martingale)
  \end{itemize}
\end{frame}

\begin{frame}{Bounding Mean First-Passage Times}{Moment Matrices and Semi-Definite Programs}
  \begin{itemize}
    \item semi-definite moment constraints (positive variance as example)
    \item hint at localizing matrices
  \end{itemize}
\end{frame}

\begin{frame}{Bounding Mean First-Passage Times}{Results and Practical Issues}
  \begin{itemize}
    \item moment stiffness, re-scaling issue
    \item some examples
  \end{itemize}
\end{frame}

\begin{frame}{Bounding Mean First-Passage Times}{Hausdorff Constraints and Linear Programs}
  \begin{itemize}
    \item linear constraints possible if domains (time and space) are finite
    \item 1D visualization of Hausdorff constraints
  \end{itemize}
\end{frame}

\begin{frame}{Linear Control Variates}{Using Correlated RVs with Known Expected Value}
  \begin{itemize}
    \item segue: use the same martingale constraints to enhance MC estimation
    \item use correlations between target RV and martingales (linear regression, i.e.\ control variates)
  \end{itemize}
\end{frame}

\begin{frame}{Linear Control Variates}{Finding Efficient Sets of Control Variates}
  \begin{itemize}
    \item time-weighting has a large influence on the correlation
    \item Infinitely many possibilities (cost needs to be controlled though)
    \item variates can be highly redundant (correlated) and incur an additional cost
    \item Alg.~1: Tighten an initial proposal set
    \item Alg.~2: Re-sample promising candidates
  \end{itemize}
\end{frame}

\begin{frame}{Linear Control Variates}{Results}
  \begin{itemize}
    \item best example?
  \end{itemize}
\end{frame}

\begin{frame}{State-Space Aggregation}{Treating Hyper-Cubes of States as One}
\end{frame}

\begin{frame}{Stationary Distribution}{Finite-Space Projection}
\end{frame}

\begin{frame}{Stationary Distribution}{Iterative Refinement Algorithm}
\end{frame}

\begin{frame}{Bridging Problem}{Dynamical Analysis Under Initial \emph{and} Terminal Constraints}
\end{frame}

\begin{frame}{Conclusions and Future Directions}
\end{frame}

\begin{frame}{Bibliography}
\end{frame}


\begin{frame}{Foster-Lyapunov Functions}
\end{frame}

\begin{frame}{Local Augmentation of Foster-Lyapunov Functions}
\end{frame}

\begin{frame}{Control Variates in General}
\end{frame}

\begin{frame}{Control Variates Selection Algorithm 1}
\end{frame}

\begin{frame}{Control Variates Selection Algorithm 2}
\end{frame}

\begin{frame}{Semi-definite programming}
\end{frame}

\end{document}
