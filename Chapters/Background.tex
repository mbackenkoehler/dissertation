\chapter{Background}

\section{Continuous-time Markov Chains}
\begin{itemize}
   \item Basic definitions
   \item Properties (non-explosivity, ergodicity, reversibility, irreducibility etc.)
\end{itemize}

\section{Markovian Population Models}
A Markov population model (MPM)
describes the stochastic interactions
among agents of distinct types in a well-stirred system.
This assumes that all agents are equally distributed in space, which
allows us to keep track only of the overall copy number of agents for each type.
Therefore the state-space is $\mathcal{S}\subseteq\mathbb{N}^{n_S}$ where
$n_S$ denotes the number of agent types or populations.
Interactions between agents are expressed as \emph{reactions}.
These reactions have associated
gains and losses of agents, given by non-negative integer vectors   
${v}_j^{-}$ and ${v}_j^{+}$ for reaction $j$, respectively. The overall change by a reaction is given by the vector $v_j = v_j^+ - v_j^-$.
A reaction between agents of types $S_1,\dots, S_{n_S}$ is specified in the following form:
\begin{equation}\label{eq:reaction}
    \sum_{\ell=1}^{n_S} v_{j\ell}^{-} S_\ell
    \xrightarrow{\alpha_j( x)}
    \sum_{\ell=1}^{n_S} v_{j\ell}^{+} S_\ell\,.
\end{equation}
The propensity function $\alpha_j$ gives the rate of the exponentially distributed firing
time of the reaction as a function of the current system state $x\in \mathcal{S}$.
In population models, \emph{mass-action} propensities are most common.
In this case the firing rate is given by the product of the number
of reactant combinations in $x$ and a
\emph{rate constant} $c_j$, i.e.
\begin{equation}\label{eq:stoch_mass_action}
    \alpha_j({x})\coloneqq c_j\prod_{\ell=1}^{n_S}\binom{x_\ell}{v_{j\ell}^{-}}\,.
\end{equation}
In this case, we give the rate constant in \eqref{eq:reaction} instead of the function $\alpha_j$.
For a given set of $n_R$ reactions, we define a stochastic
process $\{{{X}}_t\}_{t\geq 0}$ describing the evolution of the population
sizes over time $t$.
Due to the assumption of exponentially distributed firing times\marginpar{Note that in addition mild regularity assumptions
are   necessary for the existence of a unique CTMC $X$, such as non-explosiveness \cite{anderson2012continuous}.
These assumptions  are  typically
valid for realistic reaction networks.},  $ X$ is
a continuous-time
Markov chain (CTMC) on $\mathcal{S}$ with infinitesimal  generator matrix $Q$, where
the entries of $Q$ are
\begin{equation}\label{eq:cme_generator}
    Q_{ x,  y} = \begin{cases}
        \sum_{j: x+ v_j = y}\alpha_j( x)\,,&\text{if}\; x\neq
         y,\\[1ex]
        -\sum_{j=1}^{n_R} \alpha_j( x)\,, &\text{otherwise.}
    \end{cases}
\end{equation}
The probability distribution over time is given by an
initial value problem.
Given an initial state $x_0$, the distribution\marginpar{We assume an enumeration of all states in  $\mathcal{S}$. We simply write $x_i$ for the state with index $i$ and drop this notation for entries of a state $x$.  }
\begin{equation}\label{eq:forw_prob}
\pi(x_i, t)=\Pr(X_t=x_i\mid X_0=x_0),\quad t\geq 0
\end{equation}
evolves according to the Kolmogorov forward equation
\begin{equation}\label{eq:forward}
\frac{d}{dt}\pi(t) = \pi(t) Q\,,
\end{equation}
where $\pi(t)$ is an arbitrary vectorization $(\pi(x_1,t), \pi(x_2,t),\dots,\pi(x_{|\mathcal{S}|},t))$ of the states.
\eqref{eq:forw_prob} given for a single state, in the context of quantitative biology, it is commonly referred to
as the \emph{chemical master equation} (CME)
\begin{equation}\label{eq:cme}
    \frac{d\pi}{d t} ( x,t) =
    \sum_{j=1}^{n_R}\left(
        \alpha_j( x- v_j)\pi( x- v_j,t) - \alpha_j( x)\pi( x,t)
    \right)\,.
\end{equation}
A direct solution of \eqref{eq:cme} is usually not possible.
If the state-space with non-negligible probability is suitably small, a state space
truncation could be performed. That is, \eqref{eq:cme} is integrated on a possibly time-dependent subset
$\hat{\mathcal{S}}_t\subseteq\mathcal{S}$ \cite{henzinger2009sliding,munsky2006finite,spieler2014numerical}.
Instead of directly analyzing \eqref{eq:cme}, one often resorts to simulating trajectories.
A trajectory $\tau=x_0t_1x_1t_1\dots t_n x_n$ over the interval $[0,T]$ is a sequence of states $x_i$
% \vw{warum s nicht x fuer states? Eigentlich i=0 }
and corresponding
jump times $t_i$, $i=1,\dots,n$ and $t_n=T$.
We can sample trajectories of $X$ by using stochastic simulation~\cite{gillespie1977exact}.

\paragraph{Example} Consider a birth-death process as a simple example. This model is used to describe a wide variety of phenomena and often constitutes a sub-module of larger models.
For example, it represents an M/M/1 queue with service rates being linearly dependent on the queue length.
Note that even for this simple model, the state-space is countably infinite.
% \MB{could remove model environment for space}
\begin{model}[Birth-Death Process]\label{model:bd}
The model consists of exponentially distributed arrivals and service times proportional to queue length. It can be expressed using two mass-action reactions:
$$ \varnothing \xrightarrow{\mu} S \qquad\text{and}\qquad S \xrightarrow{\gamma} \varnothing\,.$$
The initial condition $X_0=0$ holds with probability one.
\end{model}

For \autoref{model:bd} the change of probability mass in a single state $x>0$ is described by expanding
\eqref{eq:cme} and
$$\frac{d}{dt}\pi_t(x)=\gamma \pi_t(x-1) + \delta \pi_t(x+1) - (\gamma + \delta)\pi_t(x)\,.$$

\section{Stochastic Simulations}
We can generate trajectories of this model by choosing either reaction, with a probability that is
proportional to its rate given the current state $x_i$ \cite{gillespie1977exact}.
The jump time $t_i- t_{i+1}$ is determined by sampling from an exponential distribution with rate $\gamma+x_i\delta$.
The simulation algorithm consists of repeatedly evaluating the race condition and jump times induced by~\eqref{eq:cme_generator}.
\begin{algorithm}
    $\tau \leftarrow$ empty list, $s\leftarrow$ sample from $\pi_0$, $t\leftarrow 0$\;
    \While{$t<T$}{
        $r_i\leftarrow\alpha_i(s),\;i=1,\dots,n_R$\;
	$k\leftarrow$ sample reaction $i$ with probability $r_i/\sum_ir_i$\;
	$dt\sim \text{Exp}\left(\sum_i r_i\right)$\;
        $s\leftarrow s + v_k$\;
	$t \leftarrow t + dt$\;
    }
    \textbf{return} $\tau$\;
    \caption{\label{alg:ssa}Sample a trajectory}
\end{algorithm}


\section{Moment Dynamics}
