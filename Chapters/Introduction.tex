\chapter{Introduction}
%\begin{center}
 %Machines were mice and men were lions\\
    %once upon a time.\\
    %But now that it's the opposite\\
    %it's twice upon a time.\\
    %\medskip
        %\emph{--- Moondog}
%\end{center}


%The scientific method revolves around the finding and testing falsifiable hypotheses.
%Such hypotheses can be simple statements such as water being liquid at room temperature and an average air pressure.
%All hypotheses involve some sort of model~--~some more well-defined than others.
Any model is an abstraction of reality.
Often this abstraction aims to capture a few aspects of interest while ignoring others.
Modelling often is a delicate balance of adequate representation and abstraction.
The former is the goal to capture all the relevant behaviours and effects in a model.
Abstractions are made for the benefit of both, explainability and facilitating analysis.
This balance between faithfully capturing reality and enabling an analysis is at the core of all modelling efforts.
%starts with a number of assumptions that simplify the model enough to reasonably try to draw conclusions.


In the natural sciences, we often deal with concentrations of different ``things''.
This could be the amount of some substance in a beaker or the number of agents waiting for some service.
Traditionally, \aclp{ODE} are the most popular paradigm for dynamical models.
\Aclp{ODE} have two inherent simplifications: Firstly, they impose a continuous state space and secondly they treat are deterministic.
The first assumption is appropriate in many circumstances.
For example, chemical concentrations can usually be treated as continuous because the number of molecules in a given volume is large enough for the concentration to be treated as a real number.
However, this simplification fails where discrete effects are central to the model.
Consider, for example the die-out of a species in the classical Lotka-Volterra predator-prey model \parencite{lotka1925elements}.
In this case the distinction between a very low concentration and a zero concentration becomes relevant again:
If the predator, e.g.\ the foxes, die out the model's dynamic changes dramatically.
Such an event can be missed by the deterministic model which instead gives a population size larger than zero but much smaller than a single individual.
\Citeauthor{mollison1991dependence} calls this problem is humorously the \emph{atto-fox} problem \parencite{mollison1991dependence}.
The second issue, i.e.\ the issue of determinism, is often closely connected to the first.
While the issue of a deterministic vs.\ a stochastic world is more an issue of philosophy and physics, stochasticity undoubtably provides an excellent abstraction for many phenomena.
Staying in with the previous example, it makes sense to consider the dying out of the predatory species to be stochastic.
This means, there is a non-zero probability for both, survival and die-out.
And answering questions about such probabilities would be one of the central functions of such a model.

The previous example falls in the realm of systems biology.
There are many more applications in systems biology that benefit from a discrete and stochastic description \parencite{wilkinson2018stochastic,BuchWolkenhauer}.
Many biological processes such as cell functions are driven by stochastic effects.
Consider for example the oscillatory processes such as the circadian cycle \parencite{asgari2019mathematical} or even classical models such as the famous predator-prey model \parencite{lotka1925elements}.
Another phenomenon central to many biological functions is a switch-like behaviour.


\Acfp{MPM} are a powerful framework to capture stochastic interactions between groups of identical agents.
Simplify representation for more efficiency per computation.
States are discrete counts of agents and reaction events stochastically change these counts.
Interactions between agents, commonly referred to as \emph{reactions}, happen at exponentially distributed random times. 
Their rate depends only on the current system state, i.e.\ the population sizes.
This kind of population model introduces the major assumption, that all agents behave similar.
Similarity entails both, similar behaviour of each agent and a homogenous mixture of all agents.\marginpar{Often the term \emph{well-stirredness} is used to express this.}
This feature is a \emph{Markov property} and
therefore \acp{MPM} form a special class of structured \acp{CTMC} \parencite{anderson2012continuous}.
Accordingly, the time-evolution of the corresponding probability distribution is 
given by the Kolmogorov equations.\marginpar{In the context of (bio-)chemistry this is often called the \emph{chemical master equation}.}
For small chains, the Kolmogorov equations can be used easily to compute accurate transition probabilities.
The \acp{CTMC} underlying most population models, however, can have large, often unbound, populations.
This makes their numerical solution extremely challenging.
Many problems occuring in a wide range of areas such as chemistry~\parencite{gillespie1977exact}, epidemiology~\parencite{mode2000stochastic},    queuing systems~\parencite{breuer2003markov}, finance~\parencite{pardoux2008markov}, and performance analysis~\parencite{bortolussi2013continuous,gast2019} can be described using this formalism.

Faced with such models, typical research questions target the forward behaviour.
This is the problem of looking into the future, given a starting distribution.
As an example consider a the question how large the probability is that a system stays in some subset of the state space for a fixed amount of time.
Mainly, we target these kinds of questions, but in \autoref{ch:bridging} we look at a closely
related class of problems -- the bridging problem.
In this problem both the initial and the terminal distribution is fixed.
\marginpar{This scenario is akin to the smoothing problem from stochastic filtering.}
Such questions are of particular interest in multimodal models to determine the switching behaviour from one to another attracting region.
A third area is the question of long-term behavior.
In \emph{ergodic} models, the forward probabilities converge to a stable equilibrium distribution which is independent of the initial distribution.\turnto{sec:stationary_dist}
%\section*{Main Research Questions}
%\paragraph{Transition Problems} How does the process get from state $A$ to state $B$?
%\paragraph{Forward Behavior} How does the process behave given an initial state?
%\paragraph{Approximations} How can we use moments to tackle the above questions?
%\paragraph{Applications} Motivate problems through multimodal behaviors.

The goal of this thesis is to develop methods to analyze \acp{MPM}.
A particular focus of these methods is scalability.
All methods developed in this thesis aim to provide means to more efficiently analyze such systems -- or at least show up paths that can lead to scalable methodologies.
Importantly, these methods are \emph{reliable} approximations: There are no ad-hoc approximations influencing estimates.
The methodological theme is to leverage (approximate) distributional information such as moment equations and state-space aggregation to obtain good approximations.

The first contribution is presented in \autoref{ch:lyapunov}.
Therein we suggest to locally alter valid proposal Lyapunov functions.
Such local alterations are subject to very few constraints.
Thus methods such as neural networks can be used.
The resulting sets can be much smaller than the ones induced by the proposal function.

In \autoref{pt:moments} we focus on moment-based techniques.
Methods to analyze such models without ignoring their inherent stochasticity has received much attention in recent years.
Facing state-spaces too large to handle using the Kolmogorov equation directly, stochastic simulations \parencite{gillespie1977exact} provide an alternative path.
Such simulations have the advantage of providing an accurate approximation of the process.
That is, they converge in the limit of simulation runs.
Stochastic simulations and estimation have been extended quite a bit.
In \autoref{ch:cvinsrns} we develop on such extension for variance reduction of stochastic estimates.
This variance reduction exploits moment dynamics to derive constraints on population averages.
These constraints are a correction to the estimate.
The corrected estimate is unbiased and has variance lower or equal to the uncorrected estimate.

Similar moment constraints can be used  --~in combination with general moment constraints~-- to formulate convex optimization problems.
This way, rigorous bounds on reaching probabilities and \aclp{MFPT} can be computed.
This approach is developed in \autoref{ch:MFPT}.
Fortunately, both of these moment-based approaches can do without any moment closures
and thereby avoid any ad-hoc approximations.
%\begin{itemize}
    %\item Moment closures, in contrast, suck because they are (mostly) not justified at all
    %\item sigma expansion
    %\item langevin equation
    %\item Moment based stuff
    %\item Hybrid methods
%\end{itemize}

In \autoref{pt:aggregation} we use a hyper cube state-space aggregation scheme to gain a rough understanding of the dynamics.
The aggregation is then further refined (\autoref{ch:statagg} and \autoref{ch:bridging}) or used to guide stochastic simulations (\autoref{ch:is}).

%aspects
%\begin{itemize}
    %\item Brownian motion is a physical motivation for memoryless property
%\end{itemize}
%Story:
%\begin{itemize}
    %\item Goal: make these models ``analyzable''
    %\item Further the effort of finding scalable methods which can analyze the stochasticity
    %\item Methods should be \emph{reliable}
    %\item All methods developed here, are or yield approximations in the pure mathematical sense
    %\item They converge to the ``true'' value
    %\item The theme connecting all approaches is to connect (approximate) distributional information such as moment equations and state-space lumping with methods yielding such approximations
%\end{itemize}
%% MFPT bounds

%This subclass of \acfp{CTMC}  is used
%to describe the stochastic dynamics of systems in various domains.
%Prominent applications are chemical reaction networks in quantitative
%biology~\parencite{BuchWolkenhauer},
%epidemic spreading~\parencite{porter2016dynamical}, performance analysis  of technical and
%information systems~\parencite{bortolussi2013continuous,gast2019} as well as the behavior of
%collective adaptive systems~\parencite{bernardo2016}.

%% Stationary aggregation

%In many areas of science, stochastic models  of interacting populations can describe systems in which the discrete population sizes evolve stochastically in continuous time.
%Such problems naturally occur in a wide range of areas such as chemistry~\parencite{gillespie1977exact}, systems biology~\parencite{wilkinson2018stochastic,BuchWolkenhauer}, epidemiology~\parencite{mode2000stochastic} as well as    queuing systems~\parencite{breuer2003markov} and finance~\parencite{pardoux2008markov}.


%% Bridging

%\Acp{MPM} are widely used to model the time evolution of complex 
%discrete phenomena in continuous time. Such problems naturally occur in a wide range of areas such as chemistry~\parencite{gillespie1977exact}, systems biology~\parencite{wilkinson2018stochastic,BuchWolkenhauer}, epidemiology~\parencite{mode2000stochastic} as well as    queuing systems~\parencite{breuer2003markov} and finance~\parencite{pardoux2008markov}.
%% The Markov property renders model analysis   feasible but requires a complete description of the current state
%% to describe the future behavior of the chain.
%In many applications, an \ac{MPM} describes the stochastic interaction of populations of agents.
%%and therefore exhibits a population structure.
%The state variables are counts of individual entities of different populations.

%% LCVs

%Markovian Population Models that are used to describe cellular processes are often subject to inherent stochasticity.
%The dynamics of gene expression, for instance, is influenced by 
%single random events (e.g.\ transcription factor binding) and 
%hence, models that take this randomness into account must monitor
%discrete molecular counts and reaction events that change these counts.
%Discrete-state continuous-time Markov chains have successfully  been
%used to describe  networks of chemical reactions
%over time that correspond to the basic events of such processes. 
%The time-evolution of the corresponding probability distribution is 
%given by the chemical master equation, whose numerical solution is
%extremely challenging because of the enormous size of the underlying
%state-space.

\section{Organization}
\autoref{fig:chap_deps} provides an overview of the dependencies between chapters.
All contributions presented in this thesis are related to \aclp{MPM}.
Therefore the background chapter, i.e.\ \autoref{ch:background} is relevant
to all later parts of the thesis.
The prerequisite for \autoref{ch:statagg} and \autoref{ch:bridging} is the aggregation
technique presented in \autoref{ch:lumping}.
Note that \autoref{ch:MFPT} and \autoref{ch:cvinsrns} share the temporal moment approach presented
in both chapters.
The conceptual connection between the bridging problem (\autoref{ch:bridging}) and the rare event sampling (\autoref{ch:is}) lies in both, its setting and the utilization of backwards probabilities.
\begin{figure}[htb]
	\centering
\begin{tikzpicture}
	\begin{scope}
  \path[mindmap,concept color=gray!25,text=black,level 1 concept/.append style =
      {sibling angle=40}]
        node[concept] {\autoref{ch:background}\\Markovian\\population\\models}
    [clockwise from=-30]
        %child[concept]{
            %node[concept] (lya) {\autoref{ch:lyapunov}\\Lyapunov functions}
        %}
    child[concept] {
	    node[concept] {\autoref{ch:lumping}\\Aggregation}
        [clockwise from=35]
	    child[concept] {
            node[concept] (stat) {\autoref{ch:statagg}\\Stationary behavior}
	    }
	    child[concept] {
            node[concept] (bridge) {\autoref{ch:bridging}\\Bridging problem}
	    }
	    child[concept] {
            node[concept] (rare) {\autoref{ch:is}\\Rare events}
	    }
    }
    child[concept] {
	    node[concept] (cv) {\autoref{ch:cvinsrns}\\Control variates}
    }
    child[concept] {
	    node[concept] (mfpt) {\autoref{ch:MFPT}\\Bounding \acsfont{MFPT}s}
    }
    ;
    \begin{pgfonlayer}{background}
        \draw [concept connection] (cv) edge (mfpt);
        \draw [concept connection] (bridge) edge (rare);
        %\draw [concept connection] (lya) edge (stat);
    \end{pgfonlayer}
	\end{scope}
 %    \begin{pgfonlayer}{background}
 %  %\draw [draw=gray!25,fill=gray!25, decorate,decoration=circle connection bar]
 %            \path (cv) to[draw=gray!25,color=gray!25,fill=gray!25, circle connection bar] (mfpt);
%%     \draw[circle connection bar]
%%%%% (cv) edge (mfpt);
 %    \end{pgfonlayer}
\end{tikzpicture}
	\caption{\label{fig:chap_deps}Chapter dependencies.}
\end{figure}


%*******************************************************
% Publications
%*******************************************************
\section{Previous Publications}%\graffito{This is just an early --~and currently ugly~-- test!}
The ideas and much of the presented results have appeared previously in the following publications.
As such, the content of most chapters have undergone peer-review and been published in various conference proceedings.
The publications and their respective sections are as indicated below.

\begin{itemize}

\item \autoref{ch:MFPT} has with minor differences been published as
\begin{quote}
    \fullcite{backenkohler2019bounding}.
\end{quote}
The approach was conceived by M.\ B.
Author M.\ B.\ performed the implementation and evaluation with feedback from the other authors.
All authors contributed to the text.

\item \autoref{ch:cvinsrns} has with minor differences been published as
\begin{quote}
    \fullcite{backenkohler2019control}.
\end{quote}
The control variate approach was conceived by M.\ B.\ and V.\ W. The refinement algorithm was developed during discussions of all authors.
Author M.\ B.\ performed the implementation and evaluation with feedback from the other authors.
All authors contributed to the text.

The contents of \autoref{sec:splitting} have been published in the article
\begin{quote}
    \fullcite{backenkohler2021variance}.
\end{quote}
The control variate approach was conceived by M.\ B.\ and V.\ W. The resampling algorithm was developed during discussions of all authors.
Author M.\ B.\ performed the implementation and evaluation with feedback from the other authors.
All authors contributed to the text.

\item \autoref{ch:statagg} has with minor differences been published as
\begin{quote}
    \fullcite{backenkohler2021abstraction}.
\end{quote}
The lumping approach was conceived by M.\ B.
All authors contributed to the text.

\item \autoref{ch:bridging} has with minor differences been published as
\begin{quote}
    \fullcite{backenkohler2020analysis}.
\end{quote}
\end{itemize}
The lumping approach was conceived by M.\ B.
All authors contributed to the text.
\autoref{ch:lumping} is in large part based on this and the latter two publication above.
\autoref{ch:background} contains introductory material and examples from all of the above publications.

%\noindent Put your publications from the thesis here. The packages \texttt{multibib} or \texttt{bibtopic} etc. can be used to handle multiple different bibliographies in your document.

% \begin{refsection}[ownpubs]
%     \small
%     \noparencite{*} % is local to to the enclosing refsection
% 
%     \printbibliography[heading=none]
% \end{refsection}


