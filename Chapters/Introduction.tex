\chapter{Introduction}
%\begin{center}
 %Machines were mice and men were lions\\
    %once upon a time.\\
    %But now that it's the opposite\\
    %it's twice upon a time.\\
    %\medskip
        %\emph{--- Moondog}
%\end{center}


The scientific method revolves around the finding and testing falsifiable hypotheses.
Such hypotheses can be simple statements such as water being liquid at room temperature and an average air pressure.
All hypotheses involve some sort of model~--~some more well-defined than others.
A model is an abstraction of the ``real world''.
Often this abstraction aims to capture a few aspects of interest while others are simplified or ignored.
Therefore a model starts with a number of assumptions that simplify the model enough to reasonably try to draw conclusions.


In the natural sciences, we often deal with concentrations of different ``things''.
This could be the amount of some substance in a beaker or the number of agents waiting for some service.



Modelling often is a delicate balance of adequate representation and abstraction.
The former is the goal to capture all the relevant behaviours and effects in a model.
Abstractions are made for the benefit of both, explainability and facilitating analysis.

Simplify representation for more efficiency per computation.

% Motivation MPMs

\begin{itemize}
    \item chemistry~\parencite{gillespie1977exact}
    \item systems biology~\parencite{wilkinson2018stochastic,BuchWolkenhauer}
    \item epidemiology~\parencite{mode2000stochastic}
    \item queuing systems~\parencite{breuer2003markov} 
    \item finance~\parencite{pardoux2008markov}
\end{itemize}


Story:
\begin{itemize}
    \item Stochasticity important
    \item Ignored, to keep models more manageable
    \item This misses important features (rare events)
    \item Invalid for lower counts
    \item Goal: make these models ``analyzable''
    \item Further the effort of finding scalable methods which can analyze the stochasticity
    \item Methods should be \emph{reliable}
    \item Moment closures suck, because they are (mostly) not justified at all
    \item All methods developed here, are or yield approximations in the pure mathematical sense
    \item They converge to the ``true'' value
    \item The theme connecting all approaches is to connect (approximate) distributional information such as moment equations and state-space lumping with methods yielding such approximations
\end{itemize}
% MFPT bounds

\Acfp{MPM} provide a
widely used framework to capture stochastic interactions between groups of identical agents.
This subclass of \acfp{CTMC}  is used
to describe the stochastic dynamics of systems in various domains.
Prominent applications are chemical reaction networks in quantitative
biology~\parencite{BuchWolkenhauer},
epidemic spreading~\parencite{porter2016dynamical}, performance analysis  of technical and
information systems~\parencite{bortolussi2013continuous,gast2019} as well as the behavior of
collective adaptive systems~\parencite{bernardo2016}.

% Stationary aggregation

In many areas of science, stochastic models  of interacting populations can describe systems in which the discrete population sizes evolve stochastically in continuous time.
Such problems naturally occur in a wide range of areas such as chemistry~\parencite{gillespie1977exact}, systems biology~\parencite{wilkinson2018stochastic,BuchWolkenhauer}, epidemiology~\parencite{mode2000stochastic} as well as    queuing systems~\parencite{breuer2003markov} and finance~\parencite{pardoux2008markov}.

Interactions between agents, commonly referred to as \emph{reactions}, happen at exponentially distributed random times. 
Their rate depends on the current system state, i.e.\ the population sizes.
This results in a continuous-time Markov chain semantics~\parencite{anderson2012continuous}.

% Bridging

\Acp{MPM} are widely used to model the time evolution of complex 
discrete phenomena in continuous time. Such problems naturally occur in a wide range of areas such as chemistry~\parencite{gillespie1977exact}, systems biology~\parencite{wilkinson2018stochastic,BuchWolkenhauer}, epidemiology~\parencite{mode2000stochastic} as well as    queuing systems~\parencite{breuer2003markov} and finance~\parencite{pardoux2008markov}.
% The Markov property renders model analysis   feasible but requires a complete description of the current state
% to describe the future behavior of the chain.
In many applications, an \ac{MPM} describes the stochastic interaction of populations of agents.
%and therefore exhibits a population structure.
The state variables are counts of individual entities of different populations.

% LCVs

Markovian Population Models that are used to describe cellular processes are often subject to inherent stochasticity.
The dynamics of gene expression, for instance, is influenced by 
single random events (e.g.\ transcription factor binding) and 
hence, models that take this randomness into account must monitor
discrete molecular counts and reaction events that change these counts.
Discrete-state continuous-time Markov chains have successfully  been
used to describe  networks of chemical reactions
over time that correspond to the basic events of such processes. 
The time-evolution of the corresponding probability distribution is 
given by the chemical master equation, whose numerical solution is
extremely challenging because of the enormous size of the underlying
state-space. 

\section{Main Research Questions}
\paragraph{Transition Problems} How does the process get from state $A$ to state $B$?
\paragraph{Forward Behavior} How does the process behave given an initial state?
\paragraph{Approximations} How can we use moments to tackle the above questions?
\paragraph{Applications} Motivate problems through multimodal behaviors.

\section{Organization}
All contributions presented in this thesis are related to \aclp{MPM}.
Therefore the background chapter, i.e.\ \autoref{ch:background} is relevant
to all later parts of the thesis.
The prerequisite for \autoref{ch:statagg} and \autoref{ch:bridging} is the aggregation
technique presented in \autoref{ch:lumping}.
Note that \autoref{ch:MFPT} and \autoref{ch:cvinsrns} share the temporal moment approach presented
in both chapters.
\autoref{fig:chap_deps} provides an overview of the dependencies between chapters.
\begin{figure}[htb]
	\centering
\begin{tikzpicture}
	\begin{scope}
  \path[mindmap,concept color=gray!25,text=black,level 1 concept/.append style =
      {sibling angle=40}]
        node[concept] {\autoref{ch:background}\\Markovian\\population\\models}
    [clockwise from=-30]
        %child[concept]{
            %node[concept] (lya) {\autoref{ch:lyapunov}\\Lyapunov functions}
        %}
    child[concept] {
	    node[concept] {\autoref{ch:lumping}\\Aggregation}
        [clockwise from=35]
	    child[concept] {
            node[concept] (stat) {\autoref{ch:statagg}\\Stationary behavior}
	    }
	    child[concept] {
		    node[concept] {\autoref{ch:bridging}\\Bridging problem}
	    }
	    child[concept] {
            node[concept] (rare) {\autoref{ch:is}\\Rare events}
	    }
    }
    child[concept] {
	    node[concept] (cv) {\autoref{ch:cvinsrns}\\Control variates}
    }
    child[concept] {
	    node[concept] (mfpt) {\autoref{ch:MFPT}\\Bounding \acsfont{MFPT}s}
    }
    ;
    \begin{pgfonlayer}{background}
        \draw [concept connection] (cv) edge (mfpt);
        \draw [concept connection] (cv) edge (rare);
        %\draw [concept connection] (lya) edge (stat);
    \end{pgfonlayer}
	\end{scope}
 %    \begin{pgfonlayer}{background}
 %  %\draw [draw=gray!25,fill=gray!25, decorate,decoration=circle connection bar]
 %            \path (cv) to[draw=gray!25,color=gray!25,fill=gray!25, circle connection bar] (mfpt);
%%     \draw[circle connection bar]
%%%%% (cv) edge (mfpt);
 %    \end{pgfonlayer}
\end{tikzpicture}
	\caption{\label{fig:chap_deps}Chapter dependencies.}
\end{figure}


%*******************************************************
% Publications
%*******************************************************
\section{Previous Publications}%\graffito{This is just an early --~and currently ugly~-- test!}
The ideas and much of the presented results have appeared previously in the following publications.
As such, the content of most chapters have undergone peer-review and been published in various conference proceedings.
The publications and their respective sections are as indicated below.

\begin{itemize}

\item \autoref{ch:MFPT} has with minor differences been published as
\begin{quote}
    \fullcite{backenkohler2019bounding}.
\end{quote}
The approach was conceived by M.\ B.
Author M.\ B.\ performed the implementation and evaluation with feedback from the other authors.
All authors contributed to the text.

\item \autoref{ch:cvinsrns} has with minor differences been published as
\begin{quote}
    \fullcite{backenkohler2019control}.
\end{quote}
The control variate approach was conceived by M.\ B.\ and V.\ W. The refinement algorithm was developed during discussions of all authors.
Author M.\ B.\ performed the implementation and evaluation with feedback from the other authors.
All authors contributed to the text.

The contents of \autoref{sec:splitting} have been published in the article
\begin{quote}
    \fullcite{backenkohler2021variance}.
\end{quote}
The control variate approach was conceived by M.\ B.\ and V.\ W. The resampling algorithm was developed during discussions of all authors.
Author M.\ B.\ performed the implementation and evaluation with feedback from the other authors.
All authors contributed to the text.

\item \autoref{ch:statagg} has with minor differences been published as
\begin{quote}
    \fullcite{backenkohler2021abstraction}.
\end{quote}
The lumping approach was conceived by M.\ B.
All authors contributed to the text.

\item \autoref{ch:bridging} has with minor differences been published as
\begin{quote}
    \fullcite{backenkohler2020analysis}.
\end{quote}
\end{itemize}
The lumping approach was conceived by M.\ B.
All authors contributed to the text.
\autoref{ch:lumping} is in large part based on this and the latter two publication above.
\autoref{ch:background} contains introductory material and examples from all of the above publications.

%\noindent Put your publications from the thesis here. The packages \texttt{multibib} or \texttt{bibtopic} etc. can be used to handle multiple different bibliographies in your document.

% \begin{refsection}[ownpubs]
%     \small
%     \noparencite{*} % is local to to the enclosing refsection
% 
%     \printbibliography[heading=none]
% \end{refsection}


