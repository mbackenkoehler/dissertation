\chapter{Introduction}


% MFPT bounds

Markovian Population Models (MPMs)    provide a
widely used framework to capture stochastic interactions between groups of identical agents.
This subclass of Continuous-Time Markov Chains (CTMCs)  is used
to describe the stochastic dynamics of systems in various domains.
Prominent applications are chemical reaction networks in quantitative
biology~\cite{BuchWolkenhauer},
epidemic spreading~\cite{porter2016dynamical}, performance analysis  of technical and
information systems~\cite{bortolussi2013,gast2019} as well as the behavior of
collective adaptive systems~\cite{bernardo2016}.







% Stationary aggregation

In many areas of science, stochastic models  of interacting populations can describe systems in which the discrete population sizes evolve stochastically in continuous time.
Such problems naturally occur in a wide range of areas such as chemistry~\cite{gillespie1977exact}, systems biology~\cite{wilkinson2018stochastic,ullah2011stochastic}, epidemiology~\cite{mode2000stochastic} as well as    queuing systems~\cite{breuer2003markov} and finance~\cite{pardoux2008markov}.

Interactions between agents, commonly referred to as \emph{reactions}, happen at exponentially distributed random times. 
Their rate depends on the current system state, i.e.\ the population sizes.
This results in a continuous-time Markov chain semantics~\cite{anderson2012continuous}.




% Bridging


Markovian Population Models  (MPM) are widely used to model the time evolution of complex  discrete phenomena in continuous time.
Such problems naturally occur in a wide range of areas such as chemistry~\cite{gillespie1977exact}, systems biology~\cite{wilkinson2018stochastic,ullah2011stochastic}, epidemiology~\cite{mode2000stochastic} as well as    queuing systems~\cite{breuer2003markov} and finance~\cite{pardoux2008markov}.
% The Markov property renders model analysis   feasible but requires a complete description of the current state
% to describe the future behavior of the chain.
In many applications, an MPM describes the stochastic interaction of populations of agents.
%and therefore exhibits a population structure.
The state variables are counts of individual entities of different populations.



% LCVs

Markovian Population Models that are used to describe cellular processes are often subject to inherent stochasticity.
The dynamics of gene expression, for instance, is influenced by 
single random events (e.g.\ transcription factor binding) and 
hence, models that take this randomness into account must monitor
discrete molecular counts and reaction events that change these counts.
Discrete-state continuous-time Markov chains have successfully  been
used to describe  networks of chemical reactions
over time that correspond to the basic events of such processes. 
The time-evolution of the corresponding probability distribution is 
given by the chemical master equation, whose numerical solution is
extremely challenging because of the enormous size of the underlying
state-space. 

\section{Organization}
\begin{itemize}
	\item Chapter dependencies graph
\end{itemize}


%*******************************************************
% Publications
%*******************************************************
\section*{Previous Publications}%\graffito{This is just an early --~and currently ugly~-- test!}
The ideas and much of the presented results have appeared previously in the following publications:

%\noindent Put your publications from the thesis here. The packages \texttt{multibib} or \texttt{bibtopic} etc. can be used to handle multiple different bibliographies in your document.

\begin{refsection}[ownpubs]
    \small
    \nocite{*} % is local to to the enclosing refsection

    \printbibliography[heading=none]
\end{refsection}


