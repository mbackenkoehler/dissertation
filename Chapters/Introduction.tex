\chapter{Introduction}



\textsf{Test}
XP
Modelling often is a delicate balance of adequate representation and abstraction.
The former is the goal to capture all the relevant behaviours and effects in a model.
Abstractions are made for the benefit of both, explainability and facilitating analysis.

Simplify representation for more efficiency per computation.


% MFPT bounds

\Acfp{MPM} provide a
widely used framework to capture stochastic interactions between groups of identical agents.
This subclass of \acfp{CTMC}  is used
to describe the stochastic dynamics of systems in various domains.
Prominent applications are chemical reaction networks in quantitative
biology~\parencite{BuchWolkenhauer},
epidemic spreading~\parencite{porter2016dynamical}, performance analysis  of technical and
information systems~\parencite{bortolussi2013continuous,gast2019} as well as the behavior of
collective adaptive systems~\parencite{bernardo2016}.







% Stationary aggregation

In many areas of science, stochastic models  of interacting populations can describe systems in which the discrete population sizes evolve stochastically in continuous time.
Such problems naturally occur in a wide range of areas such as chemistry~\parencite{gillespie1977exact}, systems biology~\parencite{wilkinson2018stochastic,ullah2011stochastic}, epidemiology~\parencite{mode2000stochastic} as well as    queuing systems~\parencite{breuer2003markov} and finance~\parencite{pardoux2008markov}.

Interactions between agents, commonly referred to as \emph{reactions}, happen at exponentially distributed random times. 
Their rate depends on the current system state, i.e.\ the population sizes.
This results in a continuous-time Markov chain semantics~\parencite{anderson2012continuous}.




% Bridging


\Acp{MPM} are widely used to model the time evolution of complex 
discrete phenomena in continuous time. Such problems naturally occur in a wide range of areas such as chemistry~\parencite{gillespie1977exact}, systems biology~\parencite{wilkinson2018stochastic,ullah2011stochastic}, epidemiology~\parencite{mode2000stochastic} as well as    queuing systems~\parencite{breuer2003markov} and finance~\parencite{pardoux2008markov}.
% The Markov property renders model analysis   feasible but requires a complete description of the current state
% to describe the future behavior of the chain.
In many applications, an \ac{MPM} describes the stochastic interaction of populations of agents.
%and therefore exhibits a population structure.
The state variables are counts of individual entities of different populations.



% LCVs

Markovian Population Models that are used to describe cellular processes are often subject to inherent stochasticity.
The dynamics of gene expression, for instance, is influenced by 
single random events (e.g.\ transcription factor binding) and 
hence, models that take this randomness into account must monitor
discrete molecular counts and reaction events that change these counts.
Discrete-state continuous-time Markov chains have successfully  been
used to describe  networks of chemical reactions
over time that correspond to the basic events of such processes. 
The time-evolution of the corresponding probability distribution is 
given by the chemical master equation, whose numerical solution is
extremely challenging because of the enormous size of the underlying
state-space. 

\section{Organization}
All contributions presented in this thesis are related to \aclp{MPM}.
Therefore the background chapter, i.e.\ \autoref{ch:background} is relevant
to all later parts of the thesis.
The prerequisite for \autoref{ch:statagg} and \autoref{ch:bridging} is the aggregation
technique presented in \autoref{ch:lumping}.
Note that \autoref{ch:MFPT} and \autoref{ch:cvinsrns} share the temporal moment approach presented
in both chapters.
\autoref{fig:chap_deps} provides an overview of the dependencies between chapters.
\begin{figure}[htb]
	\centering
\begin{tikzpicture}
	\begin{scope}
  \path[mindmap,concept color=gray!25,text=black,level 1 concept/.append style =
      {sibling angle=40}]
	node[concept] {\autoref{ch:background}\\\aclp{MPM}}
    [clockwise from=-30]
    child[concept] {
	    node[concept] {\autoref{ch:lumping}\\Aggregation}
	    [clockwise from=-20]
	    child[concept] {
		    node[concept] {\autoref{ch:statagg}\\Stationary Distribution}
	    }
	    child[concept] {
		    node[concept] {\autoref{ch:bridging}\\Bridging Problem}
	    }
    }
    child[concept] {
	    node[concept] (cv) {\autoref{ch:cvinsrns}\\Control Variates}
    }
    child[concept] {
	    node[concept] (mfpt) {\autoref{ch:MFPT}\\Bounding {MFPTs}}
    }
    ;
	\end{scope}
 %    \begin{pgfonlayer}{background}
 %  %\draw [draw=gray!25,fill=gray!25, decorate,decoration=circle connection bar]
 %            \path (cv) to[draw=gray!25,color=gray!25,fill=gray!25, circle connection bar] (mfpt);
%%     \draw[circle connection bar]
%%%%% (cv) edge (mfpt);
 %    \end{pgfonlayer}
\end{tikzpicture}
	\caption{\label{fig:chap_deps}Chapter dependencies.}
\end{figure}


%*******************************************************
% Publications
%*******************************************************
\section{Previous Publications}%\graffito{This is just an early --~and currently ugly~-- test!}
The ideas and much of the presented results have appeared previously in the following publications.
As such, most chapters have been published in various conference proceedings.
The publications and their respective sections are as indicated below.

\begin{itemize}

\item \autoref{ch:MFPT} has with minor differences been published as
\begin{quote}
	Michael Backenköhler, Luca Bortolussi, and Verena
Wolf. ``Bounding Mean First Passage Times in
	Population Con\-tin\-uo\-us-Time Markov Chains.'' In: \emph{17th
International Conference on Quantitative Evaluation of
    SysTems (\acsfont{QEST}).} Vol. 12289. \acsfont{LNCS}. Springer, 2020,
pp. 155--174.
\end{quote}

\item \autoref{ch:cvinsrns} has with minor differences been published as
\begin{quote}
	Michael Backenköhler, Luca Bortolussi, and Verena
Wolf. ``Control Variates for Stochastic Simulation of
	Chemical Reaction Networks.'' In: \emph{17th International
    Conference on Computational Methods in Systems Biology (\acsfont{CMSB}).} Vol. 11773. \acsfont{LNCS}. Springer, 2019,
pp. 42--59.
\end{quote}

\item \autoref{ch:statagg} has with minor differences been published as
\begin{quote}
	Michael Backenköhler, Luca Bortolussi, Gerrit Großmann, and Verena Wolf. ``Abstracti\-on-Guid\-ed Truncations
for Stationary Distributions of Markovian
	Population Models.'' In: \emph{18th
International Conference on Quantitative Evaluation of
    SysTems (\acsfont{QEST}).} (2021).
\end{quote}

	\item \autoref{ch:bridging} has with minor differences been published as

\begin{quote}
	Michael Backenköhler, Luca Bortolussi, Gerrit Großmann, and Verena Wolf.
	``Analysis of Markov Jump Processes under Terminal Constraints.''
	In: \emph{27th International Conference on Tools and Algorithms for
    the Construction and Analysis of Systems (\acsfont{TACAS}).}  Vol. 1265. \acsfont{LNCS}.
	Springer, 2021, pp. 210--229.
\end{quote}
\end{itemize}
\autoref{ch:lumping} is in large part based on this and the preceeding publication.
\autoref{ch:background} contains introductory material and examples from all of the above publications.

%\noindent Put your publications from the thesis here. The packages \texttt{multibib} or \texttt{bibtopic} etc. can be used to handle multiple different bibliographies in your document.

% \begin{refsection}[ownpubs]
%     \small
%     \noparencite{*} % is local to to the enclosing refsection
% 
%     \printbibliography[heading=none]
% \end{refsection}


