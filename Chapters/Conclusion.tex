\chapter{Conclusion}
\section{Contribution}
\emph{Main methodology}: The uniform structure of population models enables analysis using aggregate information (in the state domain or by using moment statistics)
\begin{itemize}
  \item Proposed novel methodologies
  \item spanning from rigorous to more heuristic
  \item hate on ad-hoc heuristics a little (i.e.\ moment closures)
  \item all methods have an approximating relationship with the actual model (convergence to the actual dynamics) -- illustrate!
\end{itemize}
\section{Future Work}
\section{The Major Challenges}
\paragraph{Practical challenge: Remembering the Domain/appplication} The main challenge in stochastic reaction networks today is the connection of methods to practicioners.
A large area body of research is devoted to methods precisely analyzing \acp{SRN}.
At times, the practical value of proposed methods is a secondary concern.
For these methods to be useful more dialog is needed with domain experts to understand the challenges.
\begin{itemize}
  \item Method and model development should be driven more by practical problems
  \item Prove methods on some domain; drive research using new possibilities
  \item Re-assess 
\end{itemize}

\paragraph{Methods at Scale}
\begin{itemize}
  \item System and state space size
\end{itemize}
