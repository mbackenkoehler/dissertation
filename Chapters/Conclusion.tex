\chapter{Conclusion}
While \acfp{MPM} are a powerful modeling framework, their analysis --~especially wrt.\ their stochasticity~-- poses a significant challenge.
This thesis has proposed a number of techniques to tackle some parts of this challenge.
These techniques have in common, that they use the uniform structure of the \acp{CTMC} underlying an \ac{MPM}.

This structure enables us to derive moment equations that yield constraints that have shown to be useful in two contexts.
In the context of mean first-passage time bounds they formed linear constraints on the moments of the process and the exit location measure.
In combination with semi-definite constraints this yields a hierarchy of convex optimization problems.
This hierarchy yields convergent bounds on the reaching probability and the mean first-passage time in a wide class of problems.
The other application of these moments constraints has been demonstrated in the context of Monte Carlo estimation:
When used as \acfp{CV} such constraints yield an alternative unbiased estimator with lower variance.
Using both heuristics and sequential Monte Carlo methods, we demonstrated that efficient constraint sets can be constructed.

The approach of aggregation takes advantage of the fact that often propensity functions result in a smooth ``landscape of transition rates'' across the state-space.
This motivates the aggregation scheme in which macro states are assumed to have a uniform distribution within.
Deriving closed-forms for the transition rates between those states, the computational load --~compared to the original model~-- can be greatly reduced.
We have demonstrated the usefulness of this aggregation scheme on the domains of stationary distributions, bridging distributions, and the use in rare event sampling.


%\section{Contribution}
%\emph{Main methodology}: The uniform structure of population models enables analysis using aggregate information (in the state domain or by using moment statistics)
%\begin{itemize}
  %\item Proposed novel methodologies
  %\item spanning from rigorous to more heuristic
  %\item hate on ad-hoc heuristics a little (i.e.\ moment closures)
  %\item all methods have an approximating relationship with the actual model (convergence to the actual dynamics) -- illustrate!
%\end{itemize}
\section{Future Work}
\paragraph{Practical challenge: Remembering the Domain/application} The main challenge in stochastic reaction networks today is the connection of methods to practicioners.
A large area body of research is devoted to methods precisely analyzing \acp{SRN}.
It is a danger that the practical value of proposed methods is a secondary concern.
For these methods to be useful more dialog is needed with domain experts to understand the challenges.
\begin{itemize}
  \item Method and model development should be driven more by practical problems
  \item Prove methods on some domain; drive research using new possibilities
  \item Re-assess 
\end{itemize}

\paragraph{Methods at Scale}
\begin{itemize}
  \item System and state space size
\end{itemize}
