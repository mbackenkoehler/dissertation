\chapter{Augmented Foster-Lyapunov Bounds}

\section{The drift and its properties}
The drift \eqref{eq:drift} plays a central role in this chapter.\marginpar{We explain the interpretation and give some basic use of the drift in \autoref{sec:statagg:lyapunov} on page~\pageref{sec:statagg:lyapunov}.}
\begin{equation}
	d(x; g) = Qg(x) = \sum_{j=1}^{n_R} \alpha_j(x) (g(x+v_j) -  g(x))
\end{equation}
An interesting fact about the drift is, that it is invariant to linear transforms to $g$.
That is
\begin{equation}
	Q(g + b)(x)
	=Qg(x)
\end{equation}
for some constant $b\in\mathbb{R}$.
Clearly a positive linear factor $m$ to $g$ factors out, i.e.\ $Q(f\circ g)(x)=mQg(x)$ for $f(x) = mx$, $m>0$.
Consequently, if the drift is scaled by its maximum value, the scaled version is invariant to linear
transforms of $g$:
\begin{equation}
	\frac{Q(f\circ g)(x)}{\max_{x\in\mathcal{S}}Q(f \circ g)(x)}
	=
	\frac{Qg(x)}{\max_{x\in\mathcal{S}}Qg(x)}
\end{equation}
Since the probability bounded sets $C_{\epsilon_{\ell}}$ depend on the scaled drift.
Therefore invariance under linear transformation implies that we cannot change --~especially improve~-- the tightness of the sets by such a transform.

\begin{example}
What makes the perfect Foster-Lyapunov function?
\marginpar{The beauty standard may vary for other applications.}
The confidence interval for level $1-\alpha$ of a Poisson with rate $\mu$ is
\[
\frac{1}{2}P^{-1}(\alpha/2; 2k)\le\mu\leq\frac{1}{2}P^{-1}(1-\alpha/2;{2k+2})\,.
\]
where $P^{-1}(\,\cdot\,; l)$ is the inverse \ac{CDF} of a $\chi^2_{l}$ distribution.
Thus, its the inverse of the regularized gamma function
\[
	\frac{1}{\Gamma(k/2)}\int_0^{x/2} t^{k/2 - 1} e^{-t}\,dt
\]
w.r.t.\ $x$.
	This gives us the ``ideal'' intervals $[l_{\epsilon},h_{\epsilon}]$, $(l_{\epsilon},h_{\epsilon})\in\mathbb{N}^2$ such that
	$\pi_{\infty}([l_{\epsilon},h_{\epsilon}])=1-\epsilon$. These sets are given as the dark area in \autoref{fig:lya_sets}.
Therefore, the perfect Lyapunov function would coincide with those sets. That is,
a function $g$ such that
\[
%	\left\{x\in\mathcal{S}\mid \frac{\epsilon}{c}Qg(x)> \epsilon - 1\right\}
	C_{\epsilon}
	=
	\left[l_{\epsilon}, h_{\epsilon}\right], \quad \forall\epsilon> 0\,.
\]
\end{example}

\section{Augmentation via Local Substitution}
We use a proved Foster-Lyapunov function as a starting point.
For many relevant reaction networks, simple choices such as L1 or L2 norms are sufficient choices \cite{spieler2014numerical}.
The resulting sets, however, are typically very large.
Tasks such as computing approximate stationary distributions on truncations set up according to these sets can be very costly.
This cost is exacerbated when a system has to be solved for a lot of different reentry matrices, which is necessary when state-wise bounds on the probability conditioned on a truncation are desired~\cite{dayar2011bounding}.

We propose to augment the proposal function by a function, that is limited to local influence guided by the initial set.
This supplementary function is phased out asymptotically using a simple sigmoid threshold function
\begin{equation}\label{eq:threshold}
  \gamma_{k,z}(x) = \frac{1}{1+k\exp(-x - y)}\,.
\end{equation}
Thus, in a one-dimensional model the augmented Lyapunov $g'$ function becomes
\begin{equation}\label{eq:thres_lyapunov}
    g'(x) = \gamma_{k,z}(x) g(x) + (1 - \gamma_{k,z}(x)) g^*(x)\,.
\end{equation}
The threshold function $\gamma$ guarantees that $g^*$ vanishes asymptotically.
The drifts $d'$ and $d^*$ are defined accordingly.

\begin{example}
Consider the example of Model~\ref{model:bd}.
Using a simple L2 norm as an initially, i.e.\ $g(x) = x^2$, we obtain a fairly large set.
In Figure~\ref{fig:bd:truncation}, we contrast this result to the solution given by the choice of $g^*(x)=(x - 250)^2$, which gives a much tighter subset with the same guarantees.
In this case, the guarantee is that the sets contain at least 0.9 stationary probability mass.
We further demonstrate how the incorporation $g^*$ into $g$ significantly tightens the set proposed initially.
\begin{figure}[htb]
    \centering
    \includegraphics[width=.6\textwidth]{gfx/lyapunov_bd.pdf}
    \caption[Manually augmented Foster-Lyapunov function]{\label{fig:bd:truncation}Different example Lyapunov functions for Model~\ref{model:bd}. The drifts $d$, $d^*$, and $d'$ are scaled and the appropriate threshold for $\epsilon=0.1$ is given.}
\end{figure}
\end{example}


The benefit of the threshold-based construction \eqref{eq:thres_lyapunov} is that we only require $g^*$ to be non-negative.
All other properties are inherited from the proposal function.
This freedom enables the use of flexible machine learning models to search an efficient $g^*$.

\begin{figure}[htb]
\centering
\includegraphics[width=\textwidth]{gfx/lya_sets.pdf}
	\caption[Augmented v.\ proposal Lyapunov sets]{\label{fig:lya_sets}Lyapunov sets for the birth-death process for different probability thresholds $\epsilon$ for the augmented function (red) and the proposal (orange). The ``perfect'' sets are computed using the confidence interval (dark).}
\end{figure}
\begin{figure}[htb]
	\centering
	\subfloat[Linear proposal]
	{\includegraphics[width=0.48\textwidth]{gfx/lnn_improvement_linear.pdf}}
	\subfloat[Quadratic proposal]
	{\includegraphics[width=0.48\textwidth]{gfx/lnn_improvement_quadratic.pdf}}
	\caption{\label{fig:improvement}}
\end{figure}

\section{Polynomial Augmentation}

\section{Neural Augmentation}
The characteristics of the augmentation function $g^*$ are typically not known beforehand.
The formulation of augmented Foster-Lyapunov functions only places basic constraints on the function used:
The function needs to be non-negative and an upper bound of the drift has to be known.
Neural networks lend themselves naturally as an extremely flexible functional family.

The central piece of fitting $g^*$ is an objective function.
Since the actual sets, bounded in probability, are defined in terms of their drift, this objective needs to be a function of this drift.
A desirable augmented drift has tight level sets with more emphasis placed on its peak regions.
A natural way to express this prioritization is the objective
\[\sum_{x\in\mathcal{S}}\int_{-\infty}^{d^*(x)}\exp(y)\,dy = \sum_{x\in\mathcal{S}} \exp(d'(x)) \]
based on the scaled augmented drift.

\begin{model}[Competitive Spread]
	Two uncoupled parallel birth-death processes (cf.~\autoref{model:par_bd})\turnto{model:par_bd} supplemented with two Gray-Scott type reactions.
%	https://groups.csail.mit.edu/mac/projects/amorphous/GrayScott/
$$\varnothing\xrightarrow{\rho} A \qquad A\xrightarrow{\delta} \varnothing \qquad
\varnothing\xrightarrow{\rho} B \qquad B\xrightarrow{\delta} \varnothing$$
$$
	2A+B\xrightarrow{\nu_1}3A\qquad
	2A+A\xrightarrow{\nu_2}3B
$$
	As a parameterization we choose $\rho = 5$, $\delta=0.1$, $\nu_1=\e{6}{-5}$, and $\nu_2=\e{6.2}{-5}$.
\end{model}
