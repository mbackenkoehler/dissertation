\chapter{Importance Sampling}
Rare events are events that occur with very low probability.
Such events can be, for example, the die-out of some population or the switching of a multimodal system across some potential barrier.
By their nature, most standard methods focus on regions with high probability.
As an example consider the standard \ac{SSA}:
Trajectories are generated according to the processes density.
Therefore unlikely events are exactly as unlikely to be sampled using the direct method.
Similarly approximations such as moment approximations or mean-field analysis focus on the main probability mass.
Therefore the analysis of such events is particularly difficult.

The arguably most used method for rare event analysis is \emph{importance sampling}.
This variance reduction method is very well-suited to the analysis of such events.
In a nutshell, this method alters the model's dynamics and keeps track of the \emph{likelihood ratio} between this altered and the original model.
This ratio provides an unbiased estimate of the event probability.
The main challenge is to find a good way to alter the model.
One popular approach is found in \acsfont{dwSSA} \parencite{kuwahara2008efficient,daigle2011automated}.
Therein each reaction rate is altered by some constant scalar factor.
These biasing values are identified by using pilot runs of the \ac{SSA} and a cross-entropy objective.
This method has been extended to be state-dependent in \citet{roh2011state}.

\section{The Bridging Master Equation}
An \ac{MPM} can be modified to fullfill terminal constraints.
As shown in \citet{huang2016reconstructing}, the bridging \ac{CME} becomes
\begin{equation}
    \frac{d\gamma}{d t} ( x,t) =
    \sum_{j=1}^{n_R}\left(
        \tilde{\alpha}_j( x- v_j)\gamma( x- v_j,t) - \tilde{\alpha}_j( x)\gamma( x,t)
    \right)\,,
\end{equation}
where the propensities
\begin{equation}
    \tilde{\alpha}_j(x, t) = \alpha_j(x)\frac{\beta(x + v_j, t)}{\beta(x, t)}
\end{equation}
depend on the backwards probabilities.

\section{Time-Dependent Simulation Algorithm}
We are changing the rate bias dynamically over fixed intervals of the time-domain.
Therefore we cannot use the default \ac{SSA}.
With \autoref{alg:ssa_dyn} we present a version of the \acl{SSA} that simulates trajectories of a system with such dynamically changing biases.
The main change is the handling of the time-discrete changes in the loop in \autoref{line:tloop}.
Here it is tested, in which time interval the sampled jkump will take place.
Therefore rates are recomputed each time the algorithm jumps forward one time-interval.
\begin{algorithm}
\SetKwInOut{Input}{input}
\SetKwInOut{Output}{output}
    \Input{$\pi_0, A$}
    \Output{trajectory $\tau$}
    $\tau \leftarrow$ empty list, $s\leftarrow$ sample from $\pi_0$, $t\leftarrow 0$, $j=1$\;
    \While{$t<T$\label{line:loop}}{
    $\tau\leftarrow \text{append}(\tau, (s, t))$\;
    $X\sim U[0,1]$\;
    $a_0\leftarrow\sum_i\alpha_{i,j}'(s)$\tcc*[r]{exit rate}
    $dt\leftarrow t_{j+1} - t$\tcc*[r]{rest of time interval}
    \While{$X>1 - \exp(-a_0 dt)$\tcc*[r]{find interval}\label{line:tloop}}{
        $X \leftarrow X - 1 + \exp(-a_0 dt)$\;
        $t\leftarrow t + dt$\;
        $j\leftarrow j+1$\;
        $dt\leftarrow $ time-interval width\;
        $a_0\leftarrow\sum_i\alpha_{i,j}'(s)$\tcc*[r]{exit rate}
    }
    $t \leftarrow t - {\log(1-X)}/{a_0}$\tcc*[r]{update time}
    $k\leftarrow$ sample reaction $i$ with probability $\alpha_{i,j}'(s)/\sum_i\alpha_{i,j}'(s)$\label{line:sample_r}\;
    $s\leftarrow s + v_k$\tcc*[r]{update state}
    }
     \textbf{return} $\tau$\;
    \caption{\label{alg:ssa_dyn}Sample a trajectory}
\end{algorithm}
